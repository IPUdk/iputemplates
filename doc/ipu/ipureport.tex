%%%%%%%%%%%%%%%%%%%%%%%%%%%%%%%%%%%%%%%%%
% IPU Example Report Document - A LaTeX Template
%
% Jorrit Wronski, jowr@ipu.dk
%%%%%%%%%%%%%%%%%%%%%%%%%%%%%%%%%%%%%%%%%
%
%
%%%%%%%%%%%%%%%%%%%%%%%%%%%%%%%%%%%%%%%%%
% SETTINGS
%%%%%%%%%%%%%%%%%%%%%%%%%%%%%%%%%%%%%%%%%
% Class options can be given as optional parameter at load time
\documentclass[short,comm,dots]{ipureport}
%\documentclass[comm,dots]{ipureport}
\usepackage[utf8]{inputenc}
\title{Design af Energimåler II for Danfoss A/S}
\subtitle{A little example on how to use the new report class}
\author{Jorrit Wronski and His Colleague}
\projectid{Projekt 7141}
\keywords{report, IPU, LaTeX, eksempel}
%\date{May 2015}
%----------------------------------------------------------------------------------------

%\covertext{Test mes sdfgsdfg sdfg sdfg sdfg sdfg sdf}
%\coverpicturewidth{0.5\textwidth}
%%\coverpicture{IPU-Logo-RGB}

%\covertext{Test mes sdfgsdfg sdfg sdfg sdfg sdfg sdf}
%\coverpicturewidth{0.5\textwidth}
%\coverpicture{IPU-Logo-RGB}

%
\usepackage[math]{blindtext}
%\usepackage[colorinlistoftodos]{todonotes} % Todonotes package for nice todos
\usepackage[backgroundcolor=ipuviridis4]{todonotes} % Todonotes package for nice todos
\usepackage[detect-all]{siunitx}

\begin{document}

%%%%%%%%%%%%%%%%%%%%%%%%%%%%%%%%%%%%%%%%%
% CONTENT
%%%%%%%%%%%%%%%%%%%%%%%%%%%%%%%%%%%%%%%%%

\selectlanguage{british}

\maketitle

%\begin{synopsis}%[Alternative Title]
%\section*{Synopsis}
%\blindtext
%\end{synopsis}


\tableofcontents
\listoftodos

%%%%%%%%%%%%%%%%%%%%%%%%%%%%%%%%%%%%%%%%%
\cleardoublepage
\section{Classes and Options}

\subsection{Colours, Logo and Papersize}

You are not encouraged to change any of the colour settings or replace the IPU logo. By default, the circle colour for the cover picture is set to the normal ipugrey and the circle background is drawn with ipugrey at \SI{10}{\percent} intensity. The same colour is also used for zebra-striped table rows, if selected. The option to define this colour is called \texttt{bgcolor}, note the American spelling. 
\todo[inline]{Implement table row colours}

To highlight certain elements, ipugreen is used. This colour is also used for the dots on the frontpage of the shoert report and the cover text in the circle for the long report. This setting can be changed with the \texttt{highlight} class option.

The IPU logo is provided as part of package, you can however change the image that is used in the top right corner by setting the option \texttt{toplogo} to a filename of any other file in the \LaTeX{} search path. 



\subsubsection{IPU 2018}
The coprorate design has changed in November 2018 and a new colour scheme has to be created: The colour scheme used for plots is based on the works of Kindlmann et al.
%\footnote{\fullcite{Kindlmann2002}}
. It is created by modifying the existing cubehelix algorithm by Green
%\footnote{\fullcite{cubehelix}}
 in the implementations of \texttt{matplotlib}
 %\footnote{\fullcite{Hunter2007}}
 . Line colours are sampled from the aforementioned maps in evenly spaced intervals skipping the lower and the upper grey-scale intensity regions denoted by the vertical bars%\footnote{Colour schemes: \\
\newcommand\hheight{0.75}\\
\makebox[\widthof{~Kindlmann et al.}]{\hfill Kindlmann et al.}:
\includegraphics[height=\hheight em]{colours/colourmap_matteoniccoli} - 
\includegraphics[height=\hheight em]{colours/colourmap_matteoniccoli_grey} (original) \\
\makebox[\widthof{~Kindlmann et al.}]{\hfill IPU colours}:
\includegraphics[height=\hheight em]{colours/colourmap_cubehelix_kindl} - 
\includegraphics[height=\hheight em]{colours/colourmap_cubehelix_kindl_grey} (adaption) \\
\makebox[\widthof{~Kindlmann et al.}]{\hfill Green}:
\includegraphics[height=\hheight em]{colours/colourmap_cubehelix_alt} - 
\includegraphics[height=\hheight em]{colours/colourmap_cubehelix_alt_grey} (original)%}.


%\definecolor{IPUdarkblue}{HTML}{#09285d}
%\definecolor{IPUdarkblue}{rgb}{(0.03707688311079281, 0.1575158489133046, 0.3665349648484976)}
%\definecolor{IPUteal}{HTML}{#00736b}
%\definecolor{IPUteal}{rgb}{(0.0, 0.45177932391406056, 0.42070350628569453)}
%\definecolor{IPUgreen}{HTML}{#1bb339}
%\definecolor{IPUgreen}{rgb}{(0.10636981679961377, 0.7030871213847766, 0.22340721209323416)}
%\definecolor{IPUlightbrown}{HTML}{#a0c53f}
%\definecolor{IPUlightbrown}{rgb}{(0.6282884950424713, 0.7739775224607538, 0.24559323885034906)}

\definecolor{name1}{rgb}{0.03707688311079281, 0.1575158489133046, 0.3665349648484976}
\definecolor{name2}{rgb}{0.0, 0.45177932391406056, 0.42070350628569453}
\definecolor{name3}{rgb}{0.10636981679961377, 0.7030871213847766, 0.22340721209323416}
\definecolor{name4}{rgb}{0.6282884950424713, 0.7739775224607538, 0.24559323885034906}

~\\
{\color[rgb]{0.03707688311079281, 0.1575158489133046, 0.3665349648484976} This text will appear in dark blue.} \\
{\color[rgb]{0.0, 0.45177932391406056, 0.42070350628569453} This text will appear in teal.} \\
{\color[rgb]{0.10636981679961377, 0.7030871213847766, 0.22340721209323416} This text will appear in green.} \\
{\color[rgb]{0.6282884950424713, 0.7739775224607538, 0.24559323885034906} This text will appear light brown.} \\

~\\
{\color{ipuviridis1} This text will appear in dark blue.} \\
{\color{ipuviridis2} This text will appear in teal.} \\
{\color{ipuviridis3} This text will appear in green.} \\
{\color{ipuviridis4} This text will appear light brown.} \\


\subsubsection{DTU Logo}

\todo[inline]{Add DTU logo option.}

The papersize used for your document can be changed with the \texttt{papersize} option. At the moment, the values \texttt{a2paper}, \texttt{a3paper}, \texttt{a4paper} and \texttt{a5paper} are supported. 

%%
\todo[inline]{[false]{crop} DeclareComplementaryOption{nocrop}{crop}}
%%
\subsection{Font Selection}
IPU uses Arial as main font. If you would like to use it too, you might have to install it. Including \texttt{\textbackslash{}usepackage\{uarial\}} gives you URW's Arial font and should work out of the box. If you have any problems with the package, try to find help online\footnote{\url{http://www.tug.org/fonts/getnonfreefonts/}}. If you do not use one of the IPU document classes, remember to set \texttt{\textbackslash{}renewcommand\{\textbackslash{}familydefault\}\{\textbackslash{}sfdefault\}} to switch to sans-serif fonts. 

Arial was created as a clone of Helvetica, which part of most Latex distributions. If you do not select anything, the classes from the template package will use Helvetica as standard font. Most people will not notice the difference and those who do might appreciate this choice. 

The boolean class options exist for the three fonts Roboto, Arial and Helvetica. You can activate the different fonts using the options \texttt{arial} or \texttt{arial=true} and similarly for \texttt{helvet} and \texttt{roboto}.Please note that the arguments are non-exclusive, but override each other if activated simultaneously possibly yielding unexpected results.

\begin{flushright}

{%\fontfamily{roboto}\selectfont 
\roboto Roboto: Aa Bb Cc Dd Ee Ff Gg Hh Ii Jj Kk Ll Mm Nn Oo Pp Qq Rr Ss Tt Uu Vv Ww Xx Yy Zz}

{\fontfamily{ua1}\selectfont Arial: Aa Bb Cc Dd Ee Ff Gg Hh Ii Jj Kk Ll Mm Nn Oo Pp Qq Rr Ss Tt Uu Vv Ww Xx Yy Zz}

{\fontfamily{phv}\selectfont Helvetica: Aa Bb Cc Dd Ee Ff Gg Hh Ii Jj Kk Ll Mm Nn Oo Pp Qq Rr Ss Tt Uu Vv Ww Xx Yy Zz}

\end{flushright}

\subsection{Header Indicators}
%
\todo[inline]{[false]{short} DeclareComplementaryOption{long}{short}}
%
\todo[inline]{[false]{conf}}
\todo[inline]{[false]{hand}}
\todo[inline]{[false]{asag}}
\todo[inline]{[false]{info}}
\todo[inline]{[false]{comm}}
%
\todo[inline]{[false]{nodots}} Overwrites the others
%
%%%%%%%%%%%%%%%%%%%%%%%%%%%%%%%%%%%%%%%%%
\section{Author Commands and Variables}

For every project, you should define \texttt{title}, \texttt{author} and \texttt{projectid}. The optional 
commands for additional information are \texttt{subtitle} and \texttt{keywords}. Use \texttt{header} to replace the default 
headline for the short reports. 

\todo[inline]{covertext{Test mes sdfgsdfg sdfg sdfg sdfg sdfg sdf}}
\todo[inline]{coverpicturewidth{0.5 textwidth}}
\todo[inline]{coverpicture{IPU-Logo-RGB}}

\todo[inline]{sent to -- option for short report missing}
\todo[inline]{options -- options for document handling of short report}

%%%%%%%%%%%%%%%%%%%%%%%%%%%%%%%%%%%%%%%%%
\blinddocument
\selectlanguage{french}
\blindmathpaper

%%%%%%%%%%%%%%%%%%%%%%%%%%%%%%%%%%%%%%%%%

\end{document}